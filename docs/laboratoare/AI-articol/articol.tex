\documentclass[12pt]{article}

\usepackage{url}
\usepackage{hyperref}

\usepackage{amsmath}

\usepackage[]{algorithm2e}

\begin{document}
	\title{Skin detection in images using machine learning techniques for color and texture recognition}
	\author{Stefan Sebastian}
	\date{\today}
	\maketitle
	
	\begin{abstract}
		The interfaces between human and computer interaction have evolved from simple buttons to voice and video recognition. Skin detection is a useful preprocessing step for detecting people in images, which facilitates this communication. This article presents a method of detecting skin in images by combining several state of the art approaches in this domain. The model combines Skin Probability Maps for color analysis on images segmented with the Quick shift algorithm and texture detection using Haralick's features. The results, evaluated on the widely used Compaq dataset, are comparable to others found in literature: 80\% true positive rate and 20\% false positive rate.
	\end{abstract}
	
	\newpage
	\tableofcontents
	\newpage
	
	
	
	
	
	\newpage
	\bibliography{references_articol}
	\bibliographystyle{ieeetr}
\end{document}